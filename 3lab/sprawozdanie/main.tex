\documentclass[12pt]{article}
\usepackage[utf8]{inputenc}
\usepackage[T1]{fontenc}
\usepackage[polish]{babel}
\usepackage{geometry}
\usepackage{tabularx}
\usepackage[table,xcdraw,dvipsnames]{xcolor}
\usepackage{color}
\usepackage{subfig}
\usepackage{sidecap}
\usepackage{wrapfig}
\usepackage{float}
\usepackage{enumerate}
\usepackage{graphicx}
\usepackage{multirow}
\usepackage{hyperref}
\usepackage{titlesec}
\usepackage{amsmath}
\usepackage{anyfontsize}
\usepackage{indentfirst}
\usepackage{listings}
\usepackage{multicol}
\usepackage{pgfplots}
\usepackage{fancyhdr}

\newgeometry{tmargin=1.8cm,bmargin=1.8cm,lmargin =1.8cm,rmargin=1.8cm}

\begin{document}

\newcommand{\zdjecie}[3]
{
    \begin{figure}[H]
        \renewcommand{\figurename}{Rys.}
        \centering
        \includegraphics[width=\textwidth]{#1}
        \caption{#2}
        \label{#3}
    \end{figure}
}

\begin{titlepage}
\begin{figure}
    \centering
    \includegraphics[width=18cm]{../../logo-PWr.png}
    \label{fig:pwr}
\end{figure}
    \begin{center}
        \LARGE \textbf{ Wydział Elektroniki, Fotoniki i Mikrosystemów }\\ 
        \vspace{70pt}
        \Huge \textit{ Sterowanie Procesami Ciągłymi}  \\
    \end{center}
    \vspace{30pt}
    \hrule
    \vspace{1pt}
    \hrule
    \begin{center}
        {\fontsize{30}{50}\selectfont Sprawozdanie nr 4\\ }
        \vspace{10pt}
        {\fontsize{25}{25}\selectfont Identyfikacja obiektu dyskretnego }
    \end{center}
    \hrule
    \vspace{1pt}
    \hrule
    \begin{flushright}
        \vspace{50pt}

        \textit{\Large Prowadzący:}\\
        \Large dr hab. inż. Grzegorz Mzyk\\
        \vspace{10pt}
        \textit{\Large Wykonała:}\\
        \Large Zuzanna Mejer, 259382 \\
        \vspace{10pt}
        \textit{\Large Termin zajęć:}\\
        \Large czwartek TP, 9:15\\
        \vspace{10pt}
    
    \end{flushright}
    \vspace{60pt}
    \begin{center}
        \large Wrocław, \today r.
    \end{center}
\end{titlepage}
    
    
\tableofcontents
\newpage

\section{Cel ćwiczenia}
Ćwiczenie było poświęcone badaniom układu automatycznej regulacji w czasie ciągłym oraz dyskretnym. Badano układy z regulatorami typu P oraz PI. Skupiono się na zależnościach między parametrami regulatorów ($k_1, k_2$) a uchybem. W przypadku badań w czasie dyskretnym, analizowano wpływ czasu próbkowania $T_d$ na uchyb.
\section{Badanie układu automatycznej regulacji w czasie ciągłym}
\subsection{Układ automatycznej regulacji z regulatorem typu P}
Dany jest obiekt regulacji o transmitancji $K_O(s) = \frac{1}{(s+1)^3}$ połączony szeregowo z regulatorem proporcjonalnym o nieznanej transmitancji $K_R(s)=k_1$. Układ $K_{OTW} = K_O(s) \cdot K_R(s)$ jest zamknięty sprzężeniem zwrotnym umożliwiającym powstanie uchybu regulacji $ \mathcal{E} = y_0(t) - y(t)$ jako sygnału wejścia na regulator. Schemat układu został przedstawiony na rys. \ref{schemat_p}.

\zdjecie{sim_p.png}{Schemat Simulink układu automatycznej regulacji z regulatorem typu P}{schemat_p}

Ten układ automatycznej regulacji jest stabilny dla $k_1 \in (-1, 8) $. Dla wybranych wartości $k_1$ z przedziału stabilności $ k_1 = [0,5 ; 2 ; 5 ; 7,5] $ narysowano charakterystyki czasowe (rys. \ref{czasowe}).

\zdjecie{reg_p.png}{Charakterystyki czasowe układu automatycznej regulacji dla wybranych wartości regulatora proporcjonalnego}{czasowe}

Dla wartości $k_1 = 5$ zauważa się, że układ ma przeregulowania na początku, później stabilizuje się na wartości około 0,82. 

\subsection{Układ automatycznej regulacji z regulatorem typu PI}
Dla wybranej wartości $k_1 = 5$ dołączono do regulatora równolegle gałąź z członem całkującym. Ta operacja ma na celu zmniejszenie uchybu.

\zdjecie{schemat_pi.png}{Schemat UAR z regulatorem typu PI}{schemat_pi}

Ten UAR jest stabilny dla $k_2 \in (0,2)$. Dla wybranych wartości $k_2 = [0,5 ; 1 ; 1,5 ; 2]$ narysowano charakterystyki czasowe (rys. \ref{char_pi}).

\zdjecie{char_pi.png}{Charakterystyki czasowe UAR z regulatorem typu PI}{char_pi}

Dla $k_2 = 2$ układ jest niestabilny. W pozostałych przypadkach dzięki dodaniu regulatora typu I, zminimalizowany został uchyb i charakterystyka czasowa stabilizuje się na wartości około 1. Im mniejsze $k_2$ tym mniejsze przeregulowania i szybsza stabilizacja układu.

\subsection{Wskaźnik jakości regulacji}
Właściwy dobór nastaw regulatora, czyli parametrów $k_1, k_2$ zagwarantuje stabilną pracę układu regulacji automatycznej oraz odpowiednią jej jakość. Jednym ze wskaźników jakości regulacji jest całka z kwadratu uchybu:

\begin{equation}
    Q = \int_{0}^{\infty} \mathcal{E} ^2 (t) \,dt 
\end{equation}
gdzie Q to wskaźnik jakości regulacji oraz $\mathcal{E}$ to uchyb. W celu zbadania wpływu wartości $k_2$ na parametr $Q$, zbudowano schemat w Simulinku (rys. \ref{jakosc_schemat}) i wygenerowano wykres zależności $Q(k_2)$ (rys. \ref{q_k2}).

\zdjecie{jakosc_schemat.png}{Schemat w Simulinku do badania kryterium jakości}{jakosc_schemat}

\zdjecie{q_k2.png}{Kryterium jakości w zależności od różnych wartości $k_2$}{q_k2}

Wartości $k_2$ były zmieniane od 0,1 do 1,9 (czyli w granicach stabilności układu) co 0,1. Należy zaznaczyć, że tym lepsza jest jakość regulacji, im mniejsze wartości osiągają wskaźniki do badania jakości regulacji. Zatem, najlepszą jakość regulacji da układ z najmniejszą wartością $k_2$ w granicach stabilności układu - z rysunku \ref{q_k2} będzie to $k_2 = 0,1$.

\section{Badanie układu automatycznej regulacji w czasie dyskretnym}
Dla dyskretnego układu automatycznej regulacji typu PI i ustalonych wartości jego nastaw $k_1 = 5$ oraz $k_2 = 0,1$ obserwowano zależność kryterium jakości Q od czasu próbkowania $T_d$. W tym celu utworzono schemat w Simulinku (rys. \ref{schemat_dyskr}). Transmitancję regulatora zapisano w dziedzinie czasu dyskretnego:
\begin{equation}
    K_R(s) = k_1 + \frac{k_2}{s} = \frac{k_1s+k_2}{s} \Rightarrow K_R(z) = \frac{k_1z+k_2}{z}
\end{equation}
\section{Podsumowanie i wnioski}


\end{document}