\documentclass[12pt]{article}
\usepackage[utf8]{inputenc}
\usepackage[T1]{fontenc}
\usepackage[polish]{babel}
\usepackage{geometry}
\usepackage{tabularx}
\usepackage[table,xcdraw,dvipsnames]{xcolor}
\usepackage{color}
\usepackage{subfig}
\usepackage{sidecap}
\usepackage{wrapfig}
\usepackage{float}
\usepackage{enumerate}
\usepackage{graphicx}
\usepackage{multirow}
\usepackage{hyperref}
\usepackage{titlesec}
\usepackage{amsmath}
\usepackage{anyfontsize}
\usepackage{indentfirst}
\usepackage{listings}
\usepackage{multicol}
\usepackage{pgfplots}
\usepackage{fancyhdr}

\newgeometry{tmargin=1.8cm,bmargin=1.8cm,lmargin =1.8cm,rmargin=1.8cm}

\begin{document}

\newcommand{\zdjecie}[3]
{
    \begin{figure}[H]
        \renewcommand{\figurename}{Rys.}
        \centering
        \includegraphics[width=\textwidth]{#1}
        \caption{#2}
        \label{#3}
    \end{figure}
}

\begin{titlepage}
\begin{figure}
    \centering
    \includegraphics[width=18cm]{../../logo-PWr.png}
    \label{fig:pwr}
\end{figure}
    \begin{center}
        \LARGE \textbf{ Wydział Elektroniki, Fotoniki i Mikrosystemów }\\ 
        \vspace{70pt}
        \Huge \textit{ Sterowanie Procesami Ciągłymi}  \\
    \end{center}
    \vspace{30pt}
    \hrule
    \vspace{1pt}
    \hrule
    \begin{center}
        {\fontsize{30}{50}\selectfont Sprawozdanie nr 4\\ }
        \vspace{10pt}
        {\fontsize{25}{25}\selectfont Identyfikacja obiektu dyskretnego }
    \end{center}
    \hrule
    \vspace{1pt}
    \hrule
    \begin{flushright}
        \vspace{50pt}

        \textit{\Large Prowadzący:}\\
        \Large dr hab. inż. Grzegorz Mzyk\\
        \vspace{10pt}
        \textit{\Large Wykonała:}\\
        \Large Zuzanna Mejer, 259382 \\
        \vspace{10pt}
        \textit{\Large Termin zajęć:}\\
        \Large czwartek TP, 9:15\\
        \vspace{10pt}
    
    \end{flushright}
    \vspace{60pt}
    \begin{center}
        \large Wrocław, \today r.
    \end{center}
\end{titlepage}
    
    
\tableofcontents
\newpage

\section{Cel ćwiczenia}
Ćwiczenie było poświęcone badaniom układu automatycznej regulacji w czasie ciągłym oraz dyskretnym. Badano układy z regulatorami typu P oraz PI. Skupiono się na zależnościach między parametrami regulatorów ($k_1, k_2$) a uchybem. W przypadku badań w czasie dyskretnym, analizowano wpływ czasu próbkowania $T_d$ na uchyb.
\section{Badanie układu automatycznej regulacji w czasie ciągłym}
\subsection{Układ automatycznej regulacji z regulatorem typu P}
Dany jest obiekt regulacji o transmitancji $K_O(s) = \frac{1}{(s+1)^3}$ połączony szeregowo z regulatorem proporcjonalnym o nieznanej transmitancji $K_R(s)=k_1$. Układ $K_{OTW} = K_O(s) \cdot K_R(s)$ jest zamknięty sprzężeniem zwrotnym umożliwiającym powstanie uchybu regulacji $ \mathcal{E} = y_0(t) - y(t)$ jako sygnału wejścia na regulator. Schemat układu został przedstawiony na rys. \ref{schemat_p}.

\zdjecie{sim_p.png}{Schemat Simulink układu automatycznej regulacji z regulatorem typu P}{schemat_p}

Ten układ automatycznej regulacji jest stabilny dla $k_1 \in (-1, 8) $. Dla wybranych wartości $k_1$ z przedziału stabilności $ k_1 = [0,5 ; 2 ; 5 ; 7,5] $ narysowano charakterystyki czasowe (rys. \ref{czasowe}).

\zdjecie{reg_p.png}{Charakterystyki czasowe układu automatycznej regulacji dla wybranych wartości regulatora proporcjonalnego}{czasowe}

Na przykład, dla wartości $k_1 = 5$ zauważa się, że układ ma przeregulowania na początku, później stabilizuje się na wartości około 0,82. 

\subsection{Układ automatycznej regulacji z regulatorem typu PI}
Dla wybranej wartości $k_1 = 5$ dołączono do regulatora równolegle gałąź z członem całkującym. Ta operacja ma na celu zmniejszenie uchybu.

\zdjecie{schemat_pi.png}{Schemat UAR z regulatorem typu PI}{schemat_pi}

Ten UAR jest stabilny dla $k_2 \in (0,2)$. Dla wybranych wartości $k_2 = [0,5 ; 1 ; 1,5 ; 2]$ narysowano charakterystyki czasowe (rys. \ref{char_pi}).

\zdjecie{char_pi.png}{Charakterystyki czasowe UAR z regulatorem typu PI}{char_pi}

Dla $k_2 = 2$ układ jest niestabilny. W pozostałych przypadkach dzięki dodaniu regulatora typu I, zminimalizowany został uchyb i charakterystyka czasowa stabilizuje się na wartości około 1. Im mniejsze $k_2$ tym mniejsze przeregulowania i szybsza stabilizacja układu.

\subsection{Wskaźnik jakości regulacji}
Właściwy dobór nastaw regulatora, czyli parametrów $k_1, k_2$ zagwarantuje stabilną pracę układu regulacji automatycznej oraz odpowiednią jej jakość. Jednym ze wskaźników jakości regulacji jest całka z kwadratu uchybu:

\begin{equation}
    Q = \int_{0}^{\infty} \mathcal{E} ^2 (t) \,dt 
\end{equation}
gdzie Q to wskaźnik jakości regulacji oraz $\mathcal{E}$ to uchyb. W celu zbadania wpływu wartości $k_2$ na parametr $Q$, zbudowano schemat w Simulinku (rys. \ref{jakosc_schemat}) i wygenerowano wykres zależności $Q(k_2)$ (rys. \ref{q_k2}).

\zdjecie{jakosc_schemat.png}{Schemat w Simulinku do badania kryterium jakości}{jakosc_schemat}

\zdjecie{q_k2.png}{Kryterium jakości w zależności od różnych wartości $k_2$}{q_k2}

Wartości $k_2$ były zmieniane od 0,1 do 1,9 (czyli w granicach stabilności układu) co 0,1. Należy zaznaczyć, że tym lepsza jest jakość regulacji, im mniejsze wartości osiągają wskaźniki do badania jakości regulacji. Zatem, najlepszą jakość regulacji da układ z najmniejszą wartością $k_2$ w granicach stabilności układu - z rysunku \ref{q_k2} będzie to $k_2 = 0,1$.

\section{Badanie układu automatycznej regulacji w czasie dyskretnym}
\subsection{Odpowiedź UAR w zależności od czasu próbkowania}
Dla dyskretnego układu automatycznej regulacji typu PI i ustalonych wartości jego nastaw obserwowano wpływ różnych czasów próbkowania na charakterystykę układu dyskretnego. W tym celu utworzono schemat w Simulinku (rys. \ref{schemat_dyskr}). 

\zdjecie{schemat_Td.png}{Schemat w Simulinku do badania układu dyskretnego}{schemat_dyskr}

Transmitancję regulatora zapisano w dziedzinie czasu dyskretnego: $ \frac{a \cdot z +b}{z-1}$, gdzie wartości $a, b$ zostały wyznaczone przez funkcję $c2d$. Badania zostały przeprowadzone dla wartości regulatora $k_1 = 1$ oraz $k_2 = 0.5$, od których zależy wartość regulatora dyskretnego. Na początku obserwowano charakterystykę czasową UAR dla różnych czasów próbkowania $Td$ (rys. \ref{rozne_Td}).

\zdjecie{rozne_Td.png}{Charakterystyka czasowa UAR dla różnych czasów próbkowania}{rozne_Td}

Z badania wynika, że optymalne czasy próbkowania to na przykład $Td = 0,3$, $Td = 0,5$ czy $Td= 0,8$. Dla większych wartości $Td$ zauważa się większe przeregulowania, podczas gdy czas stabilizacji jest porównywalny. Natomiast dla mniejszych wartości $Td$ czas stabilizacji znacznie się wydłuża.

\subsection{Wskaźnik jakości regulacji}
Następnie przeprowadzono badanie wskaźnika jakości regulacji, jakim jest całka z kwadratu uchybu. Poniższe zdjęcie przedstawia zależność $Q(Td)$ dla różnych czasów próbkowania (rys. \ref{q(td)}).


\zdjecie{Q(Td).png}{Zależność kryterium jakości Q od czasu próbkowania $T_d$}{q(td)}

Badanie potwierdza, że optymalne czasy próbkowania (zapewniające najlepszą jakość regulacji) mieszczą się w zakresie $Td = [0.4, 1.3]$. Zarówno mniejsze jak i większe wartości czasu próbkowania gwarantują gorszą jakość regulacji.

\section{Podsumowanie i wnioski}
Po przeprowadzeniu badań nad układem automatycznej regulacji wywnioskowano, że:
\begin{itemize}
    \item regulator typu I dołączony równolegle do regulatora proporcjonalnego pozwala zmniejszyć uchyb
    \item całka z kwadratu uchybu pozwala wywnioskować jakość regulacji dla zadanych parametrów
    \item parametr $k_2$ ma wpływ na przeregulowania i czas stabilizacji układu
    \item im mniejszy parametr $k_2$ w regulatorze całkującym, tym mniejsza wartość $Q$ (wskaźnika regulacji), czyli lepsza jakość regulacji
    \item czas próbkowania ma wpływ na odpowiedź UAR z regulatorem w czasie dyskretnym
    \item optymalne czasy próbkowania dla przeprowadzonych badań mieszczą się w zakresie $Td = [0.3, 0.8]$
    \item dla wartości spoza tego zakresu zauważalne były większe przeregulowania lub dłuższy czas stabilizacji
    \item wskaźnik jakości regulacji potwierdził wartości czasu próbkowania gwarantujące najlepszą jakość regulacji.
\end{itemize}


\end{document}